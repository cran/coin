
\documentclass[letter]{article}
\usepackage{amstext}
\usepackage{amsfonts}
\usepackage{hyperref}
\usepackage[round]{natbib}
\usepackage{hyperref}

%%\VignetteIndexEntry{coin: A Computational Framework for Conditional Inference}

\usepackage{verbatim}
\usepackage{graphicx}
\usepackage{amsfonts}
\usepackage{amstext}
\usepackage{amsmath}
%%\usepackage{amsthm}
%%\usepackage[round]{natbib}
%%\usepackage{bibentry}
%%\usepackage{hyperref}
%%\usepackage{thumbpdf}
\usepackage{rotating}
%%\usepackage{floatflt}

\newcommand{\R}{\mathbb{R} }
\newcommand{\Prob}{\mathbb{P} }
\newcommand{\N}{\mathbb{N} }
\newcommand{\C}{\mathbb{C} }
\newcommand{\V}{\mathbb{V}} %% cal{\mbox{\textnormal{Var}}} }
\newcommand{\E}{\mathbb{E}} %%mathcal{\mbox{\textnormal{E}}} }
\newcommand{\Var}{\mathbb{V}} %%mathcal{\mbox{\textnormal{Var}}} }
\newcommand{\argmin}{\operatorname{argmin}\displaylimits}
\newcommand{\argmax}{\operatorname{argmax}\displaylimits}
\newcommand{\LS}{\mathcal{L}_n}
\newcommand{\TS}{\mathcal{T}_n}
\newcommand{\LSc}{\mathcal{L}_{\text{comb},n}}
\newcommand{\LSbc}{\mathcal{L}^*_{\text{comb},n}}
\newcommand{\F}{\mathcal{F}}
\newcommand{\A}{\mathcal{A}}
\newcommand{\yn}{y_{\text{new}}}
\newcommand{\z}{\mathbf{z}}
\newcommand{\X}{\mathbf{X}}
\newcommand{\Y}{\mathbf{Y}}
\newcommand{\sX}{\mathcal{X}}
\newcommand{\sY}{\mathcal{Y}}
\newcommand{\T}{\mathbf{T}}
\newcommand{\x}{\mathbf{x}}
\renewcommand{\a}{\mathbf{a}}
\newcommand{\xn}{\mathbf{x}_{\text{new}}}
\newcommand{\y}{\mathbf{y}}
\newcommand{\w}{\mathbf{w}}
\newcommand{\ws}{\mathbf{w}_\cdot}
\renewcommand{\t}{\mathbf{t}}
\newcommand{\M}{\mathbf{M}}
\renewcommand{\vec}{\text{vec}}
\newcommand{\B}{\mathbf{B}}
\newcommand{\K}{\mathbf{K}}
\newcommand{\W}{\mathbf{W}}
\newcommand{\D}{\mathbf{D}}
\newcommand{\I}{\mathbf{I}}
\newcommand{\bS}{\mathbf{S}}
\newcommand{\cellx}{\pi_n[\x]}
\newcommand{\partn}{\pi_n(\mathcal{L}_n)}
\newcommand{\err}{\text{Err}}
\newcommand{\ea}{\widehat{\text{Err}}^{(a)}}
\newcommand{\ecv}{\widehat{\text{Err}}^{(cv1)}}
\newcommand{\ecvten}{\widehat{\text{Err}}^{(cv10)}}
\newcommand{\eone}{\widehat{\text{Err}}^{(1)}}
\newcommand{\eplus}{\widehat{\text{Err}}^{(.632+)}}
\newcommand{\eoob}{\widehat{\text{Err}}^{(oob)}}

\hyphenation{Qua-dra-tic}


\hypersetup{%
  pdftitle = {coin: A Computational Framework for Conditional Inference},
  pdfsubject = {Manuscript},
  pdfauthor = {Torsten Hothorn, Kurt Hornik, 
               Mark van de Wiel and Achim Zeileis},
%% change colorlinks to false for pretty printing
  colorlinks = {true},
  linkcolor = {blue},
  citecolor = {blue},
  urlcolor = {red},
  hyperindex = {true},
  linktocpage = {true},
}


\usepackage{/usr/local/lib/R/share/texmf/Sweave}
\begin{document}

\title{\texttt{coin}: A Computational Framework for Conditional Inference}
\author{Torsten Hothorn$^1$, Kurt Hornik$^2$, Mark van de Wiel$^3$ \\
        and Achim Zeileis$^2$}
\date{}
\maketitle

\noindent$^1$Institut f\"ur Medizininformatik, Biometrie und Epidemiologie\\
     Friedrich-Alexander-Universit\"at Erlangen-N\"urnberg\\
     Waldstra{\ss}e 6, D-91054 Erlangen, Germany \\
     \texttt{Torsten.Hothorn@R-project.org}
\newline

\noindent$^2$Institut f\"ur Statistik und Mathematik,
             Wirtschaftsuniversit\"at Wien \\
       Augasse 2-6, A-1090 Wien, Austria \\
       \texttt{Kurt.Hornik@R-project.org} \\
       \texttt{Achim.Zeileis@R-project.org} 
\newline

\noindent$^3$ Department of Mathematics and Computer Science \\
              Eindhoven University of Technology \\
              HG 9.25, P.O. Box 513 \\
              5600 MB Eindhoven, The Netherlands \\
              \texttt{markvdw@win.tue.nl}
\newline


\section{Introduction}

The \texttt{coin} package implements a unified approach for conditional
inference procedures commonly known as \textit{permutation tests}.
The theoretical basis of design and implementation is the unified 
framework for permutation tests given by \cite{StrasserWeber1999}. For a
very flexible formulation of multivariate linear statistics,
\cite{StrasserWeber1999} derived the conditional expectation and covariance
of the conditional (permutation) distribution as well as the multivariate 
limiting distribution. 


Conditional counterparts of a large amount of classical (unconditional) 
test procedures for 
continuous, categorical and censored data are part of this framework, for
example the Cochran-Mantel-Haenszel test for independence in general
contingency tables, linear association tests for ordered categorical 
data, linear rank tests and multivariate permutation tests.

The conceptual framework of permutation tests by \cite{StrasserWeber1999}
for arbitrary problems is available via the generic
\texttt{independence\_test}. Because convenience functions for the most
prominent problems are available, users will not have to use this
extremely flexible procedure. Currently, the conditional variants of the 
following test procedures are available: 
\begin{center}
\begin{tabular}{ll}
\texttt{oneway\_test} & two- and $K$-sample permutation test \\
\texttt{wilcox\_test} & Wilcoxon-Mann-Whitney rank sum test \\
\texttt{normal\_test} & van der Waerden normal quantile test \\
\texttt{median\_test} & Median test \\
\texttt{kruskal\_test} & Kruskal-Wallis test \\
\texttt{ansari\_test} & Ansari-Bradley test \\
\texttt{fligner\_test} & Fligner-Killeen test \\
\texttt{chisq\_test} & Pearson's $\chi^2$ test \\
\texttt{cmh\_test} & Cochran-Mantel-Haenszel test \\
\texttt{lbl\_test} & linear-by-linear association test \\
\texttt{surv\_test} & two- and $K$-sample logrank test \\
\texttt{maxstat\_test} & maximally selected statistics) \\
\texttt{spearman\_test} & Spearman's test \\
\texttt{friedman\_test} & Friedman test \\
\texttt{wilcoxsign\_test} & Wilcoxon-Signed-Rank test \\
\texttt{mh\_test} &  marginal homogeneity test. \\
\end{tabular}
\end{center}

Those convenience functions essentially perform a certain transformation of
the data, e.g., a rank transformation, and call \texttt{independence\_test}
for the computation of linear statistics, expectation and covariance and the
test statistic as well as 
their null distribution. The exact null distribution can
be approximated either by the asymptotic distribution or via conditional
Monte-Carlo for all test procedures, the exact null distribution is
available for special cases. Moreover, all test procedures allow for the
specification of blocks for stratification.

\section{Permutation Tests}

In the following we assume that we are provided with $n$ observations 
\begin{eqnarray*}
(\Y_i, \X_i, w_i, b_i), \quad i = 1, \dots, n.
\end{eqnarray*}
The variables $\Y$ and $\X$ from sample spaces $\mathcal{Y}$ and
$\mathcal{X}$ may
be measured at arbitrary scales and may be multivariate as well. In addition
to those measurements, case weights $w$ and a factor $b$ coding blocks may
be available. For the sake of simplicity, we assume $w_i = 1$ and $b_i = 0$
for all observations $i = 1, \dots, n$ for the moment. 

We are interested in testing the null hypothesis of independence of $\Y$ and
$\X$
\begin{eqnarray*}
H_0: D(\Y | \X) = D(\Y)
\end{eqnarray*}
against arbitrary alternatives. \cite{StrasserWeber1999} suggest to derive
scalar test statistics for testing $H_0$ from multivariate linear statistics
of the form 
\begin{eqnarray} \label{linstat}
\T = \vec\left(\sum_{i = 1}^n w_i g(\X_i) h(\Y_i, (\Y_1, \dots, \Y_n))^\top\right)
\in \R^{pq}.
\end{eqnarray}
Here, $g: \mathcal{X} \rightarrow \R^{p}$ is a transformation of
the $\X$ measurements and the \emph{influence function}
$h: \mathcal{Y} \times \mathcal{Y}^n \rightarrow
\R^q$ depends on the responses $(\Y_1, \dots, \Y_n)$ in a permutation
symmetric way. We will give specific examples how to choose $g$ and $h$
later on.

The distribution of $\T$  depends on the joint
distribution of $\Y$ and $\X$, which is unknown under almost all practical
circumstances. At least under the null hypothesis one can dispose of this
dependency by fixing $\X_1, \dots, \X_n$ and conditioning on all possible
permutations $S$ of the responses $\Y_1, \dots, \Y_n$. 
This principle leads to test procedures known
as \textsl{permutation tests}. 

The conditional expectation $\mu \in \R^{pq}$ and covariance 
$\Sigma \in \R^{pq \times pq}$ 
of $\T$ under $H_0$ given
all permutations $\sigma \in S$ of the responses are derived by
\cite{StrasserWeber1999}:
\begin{eqnarray}
\mu & = & \E(\T | S) = \vec \left( \left( \sum_{i = 1}^n w_i g(\X_i) \right) \E(h | S)^\top
\right), \nonumber \\
\Sigma & = & \V(\T | S) \nonumber \\
& = &
    \frac{\ws}{\ws - 1}  \V(h | S) \otimes
        \left(\sum_i w_i  g(\X_i) \otimes w_i  g(\X_i)^\top \right)
\label{expectcovar}
\\
& - & \frac{1}{\ws - 1}  \V(h | S)  \otimes \left(
        \sum_i w_i g(\X_i) \right)
\otimes \left( \sum_i w_i g(\X_i)\right)^\top
\nonumber
\end{eqnarray}
where $\ws = \sum_{i = 1}^n w_i$ denotes the sum of the case weights,
and $\otimes$ is the Kronecker product. The conditional expectation of the
influence function is
\begin{eqnarray*}
\E(h | S) = \ws^{-1} \sum_i w_i h(\Y_i, (\Y_1, \dots, \Y_n)) \in
\R^q
\end{eqnarray*}
with corresponding $q \times q$ covariance matrix
\begin{eqnarray*}
\V(h | S) = \ws^{-1} \sum_i w_i \left(h(\Y_i, (\Y_1, \dots, \Y_n))
- \E(h | S)
\right) \\
\left(h(\Y_i, (\Y_1, \dots, \Y_n)) - \E(h | S)\right)^\top.
\end{eqnarray*}

Having the conditional expectation and covariance at hand we are able to
standardize a linear statistic $\T \in \R^{pq}$ of the form
(\ref{linstat}). Univariate test statistics~$c$ mapping an observed linear
statistic $\t \in
\R^{pq}$ into the real line can be of arbitrary form.  An obvious choice is
the maximum of the absolute values of the standardized linear statistic
\begin{eqnarray*}
c_\text{max}(\t, \mu, \Sigma)  = \max \left| \frac{\t -
\mu}{\text{diag}(\Sigma)^{1/2}} \right|
\end{eqnarray*}
utilizing the conditional expectation $\mu$ and covariance matrix
$\Sigma$. The application of a quadratic form $c_\text{quad}(\t, \mu,
\Sigma)  =
(\t - \mu) \Sigma^+ (\t - \mu)^\top$ is one alternative, although
computationally more expensive because the Moore-Penrose 
inverse $\Sigma^+$ of $\Sigma$ is involved.

The conditional distribution and thus the $P$-value
of the statistics $c(\t, \mu, \Sigma)$ can be
computed in several different ways. For some special forms of the
linear statistic, the exact distribution of the test statistic is trackable.
For two-sample problems, the shift-algorithm by \cite{axact-dist:1986} 
and \cite{exakte-ver:1987} and the split-up algorithm by 
\cite{vdWiel2001} are implemented as part of the package.
Conditional Monte-Carlo procedures can be used to approximate the exact
distribution. \cite{StrasserWeber1999} proved (Theorem 2.3) that the   
conditional distribution of linear statistics $\T$ with conditional    
expectation $\mu$ and covariance $\Sigma$ tends to a multivariate normal
distribution with parameters $\mu$ and $\Sigma$ as $n, s \rightarrow
\infty$. Thus, the asymptotic conditional distribution of test statistics of
the
form $c_\text{max}$ is normal and
can be computed directly in the univariate case ($pq = 1$)
or approximated by means of quasi-randomized Monte-Carlo  
procedures in the multivariate setting \citep{numerical-:1992}. For
quadratic forms
$c_\text{quad}$ which follow a $\chi^2$ distribution with degrees of freedom 
given by the rank of $\Sigma$ \citep[Theorem 6.20, ][]{Rasch1995}, exact
probabilities can be computed efficiently.

\section{Illustrations and Applications}

The main workhorse \texttt{independence\_test} essentially allows for the
specification of $\Y, \X$ and $b$ through a formula interface of the form
\verb/y ~ x | b/, weights can be defined by a formula with one variable on
the right hand side only. Four additional arguments are available for the
specification of the transformation $g$ (\texttt{xtrans}), the influence 
function $h$ (\texttt{ytrans}), the form of the test statistic $c$ 
(\texttt{teststat}) and the null distribution (\texttt{distribution}).

\paragraph{Independent $K$-Sample Problems.}

When we want to compare the distribution of an univariate qualitative response 
$\Y$ in $K$ groups given by a factor $\X$ at $K$ levels, the transformation 
$g$ is the dummy matrix coding the groups and $h$ is either the identity
transformation or a some form of rank transformation.

For example, the Kruskal-Wallis test may be computed as follows
\citep[example taken from][Table 6.3, page 200]{HollanderWolfe1999}:

\begin{Schunk}
\begin{Sinput}
> library(coin)
\end{Sinput}
\begin{Soutput}
Loading required package: survival 
Loading required package: splines 
Loading required package: mvtnorm 
\end{Soutput}
\begin{Sinput}
> YOY <- data.frame(length = c(46, 28, 46, 37, 32, 
+     41, 42, 45, 38, 44, 42, 60, 32, 42, 45, 58, 27, 
+     51, 42, 52, 38, 33, 26, 25, 28, 28, 26, 27, 27, 
+     27, 31, 30, 27, 29, 30, 25, 25, 24, 27, 30), 
+     site = factor(c(rep("I", 10), rep("II", 10), 
+         rep("III", 10), rep("IV", 10))))
> it <- independence_test(length ~ site, data = YOY, 
+     ytrafo = function(data) trafo(data, numeric_trafo = rank), 
+     teststat = "quadtype")
> it
\end{Sinput}
\begin{Soutput}
	Asymptotical General Independence Test

data:  length by groups I, II, III, IV 
T = 22.8524, df = 3, p-value = 4.335e-05
\end{Soutput}
\end{Schunk}
The linear statistic $\T$ is the sum of the ranks in each group and 
can be extracted via
\begin{Schunk}
\begin{Sinput}
> statistic(it, "linear")
\end{Sinput}
\begin{Soutput}
    [,1]
I    278
II   307
III  119
IV   116
\end{Soutput}
\end{Schunk}
Note that \texttt{statistic(..., "linear")} currently returns the linear
statistic in matrix form, i.e.
\begin{eqnarray*}
\sum_{i = 1}^n w_i g(\X_i) h(\Y_i, (\Y_1, \dots, \Y_n))^\top \in \R^{p
\times q}.
\end{eqnarray*}
The conditional expectation and covariance are available from
\begin{Schunk}
\begin{Sinput}
> expectation(it)
\end{Sinput}
\begin{Soutput}
    [,1]
I    205
II   205
III  205
IV   205
\end{Soutput}
\begin{Sinput}
> covariance(it)
\end{Sinput}
\begin{Soutput}
          [,1]      [,2]      [,3]      [,4]
[1,] 1019.0385 -339.6795 -339.6795 -339.6795
[2,] -339.6795 1019.0385 -339.6795 -339.6795
[3,] -339.6795 -339.6795 1019.0385 -339.6795
[4,] -339.6795 -339.6795 -339.6795 1019.0385
\end{Soutput}
\end{Schunk}
and the standardized linear statistic $(\T - \mu)\text{diag}(\Sigma)^{-1/2}$
is
\begin{Schunk}
\begin{Sinput}
> statistic(it, "standardized")
\end{Sinput}
\begin{Soutput}
         [,1]
I    2.286797
II   3.195250
III -2.694035
IV  -2.788013
\end{Soutput}
\end{Schunk}
Since a quadratic form of the test statistic was requested via
\texttt{teststat = "quadtype"}, the test statistic is 
\begin{Schunk}
\begin{Sinput}
> statistic(it)
\end{Sinput}
\begin{Soutput}
[1] 22.85242
\end{Soutput}
\end{Schunk}
By default, the asymptotic distribution of the test statistic is computed,
the $p$-value is
\begin{Schunk}
\begin{Sinput}
> pvalue(it)
\end{Sinput}
\begin{Soutput}
[1] 4.334659e-05
\end{Soutput}
\end{Schunk}

Life is much simpler with convenience functions very similar to those
available in package \texttt{stats} for a long time. The exact
null distribution of the Kruskal-Wallis test can be approximated by $9999$
Monte-Carlo replications via
\begin{Schunk}
\begin{Sinput}
> kw <- kruskal_test(length ~ site, data = YOY, distribution = "approx", 
+     B = 9999)
> kw
\end{Sinput}
\begin{Soutput}
	Approximative Kruskal-Wallis Test

data:  length by groups I, II, III, IV 
T = 22.8524, p-value < 2.2e-16
\end{Soutput}
\end{Schunk}
with $p$-value (and $99\%$ confidence interval) of 
\begin{Schunk}
\begin{Sinput}
> pvalue(kw)
\end{Sinput}
\begin{Soutput}
[1] 0
99 percent confidence interval:
 0.0000000000 0.0005297444 
\end{Soutput}
\end{Schunk}
Of course it is possible to choose a $c_\text{max}$ type test statistic
instead of a quadratic form.

\paragraph{Independence in Contingency Tables.}

Independence in general two- or three-dimensional contingency tables can be
tested by the Cochran-Mantel-Haenszel test. Here, both $g$ and $h$ are dummy
matrices \citep[example data from][Table 7.8, page 288]{Agresti2002}:

\begin{Schunk}
\begin{Sinput}
> data(jobsatisfaction, package = "coin")
> it <- cmh_test(jobsatisfaction)
> it
\end{Sinput}
\begin{Soutput}
	Asymptotical Generalised Cochran-Mantel-Haenszel Test

data:  Job.Satisfaction by
	 groups <5000, 5000-15000, 15000-25000, >25000 
	 stratified by Gender 
T = 10.2001, df = 9, p-value = 0.3345
\end{Soutput}
\end{Schunk}

The standardized contingency table allowing for an inspection of the
deviation from the null hypothesis of independence of income and
jobsatisfaction (stratified by gender) is 
\begin{Schunk}
\begin{Sinput}
> statistic(it, "standardized")
\end{Sinput}
\begin{Soutput}
            Very Dissatisfied A Little Dissatisfied
<5000               1.3112789            0.69201053
5000-15000          0.6481783            0.83462550
15000-25000        -1.0958361           -1.50130926
>25000             -1.0377629           -0.08983052
            Moderately Satisfied Very Satisfied
<5000                 -0.2478705     -0.9293458
5000-15000             0.5175755     -1.6257547
15000-25000            0.2361231      1.4614123
>25000                -0.5946119      1.2031648
\end{Soutput}
\end{Schunk}

\paragraph{Ordered Alternatives.}

Of course, both job satisfaction and income are ordered variables. 
When $\Y$ is measured at $J$ levels and $\X$ at $K$ levels, 
$\Y$ and $\X$ are associated with score vectors $\xi \in
\R^J$ and $\gamma \in \R^K$, respectively. The linear statistic is now a linear
combination of the linear statistic $\T$ of the form
\begin{eqnarray*}
\M \T & = & \vec \left( \sum_{i=1}^n w_i \gamma^\top g(\X_i)
            \left(\xi^\top h(\Y_i, (\Y_1, \dots, \Y_n)\right)^\top \right)
\in \R \text{ with } \M = \xi \otimes \gamma.
\end{eqnarray*}
By default, scores are $\xi = 1, \dots, J$ and $\gamma = 1, \dots, K$.
\begin{Schunk}
\begin{Sinput}
> lbl_test(jobsatisfaction)
\end{Sinput}
\begin{Soutput}
	Asymptotical Linear-by-Linear Association Test

data:  Job.Satisfaction (ordered) by
	 groups <5000 < 5000-15000 < 15000-25000 < >25000 
	 stratified by Gender 
T = 6.6235, df = 1, p-value = 0.01006
\end{Soutput}
\end{Schunk}
The scores $\xi$ and $\gamma$ can be specified to the linear-by-linear
association test via a list those names correspond to the variable names
\begin{Schunk}
\begin{Sinput}
> lbl_test(jobsatisfaction, scores = list(Job.Satisfaction = c(1, 
+     3, 4, 5), Income = c(3, 10, 20, 35)))
\end{Sinput}
\begin{Soutput}
	Asymptotical Linear-by-Linear Association Test

data:  Job.Satisfaction (ordered) by
	 groups <5000 < 5000-15000 < 15000-25000 < >25000 
	 stratified by Gender 
T = 6.1563, df = 1, p-value = 0.01309
\end{Soutput}
\end{Schunk}

\section{Quality Assurance}

The test procedures implemented in package \texttt{coin} are continuously checked
against results obtained by the corresponding implementations in package
\texttt{stats} (where available). In addition, the test statistics and 
exact, approximative and asymptotic $p$-values for data examples given in the 
\texttt{StatXact-6} 
user manual \citep{StatXact6} are compared with the results reported in the
\texttt{StatXact-6} manual. For details on the test procedures we refer to
the \textsf{R}-transcript files in directory \texttt{coin/tests}.


\bibliographystyle{plainnat}
\bibliography{coinrefs}

\end{document}

